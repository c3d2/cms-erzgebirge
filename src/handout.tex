\documentclass[a5paper]{scrartcl}

\usepackage{hyperref}

\usepackage[german]{babel}
\usepackage[T1]{fontenc}
\usepackage[utf8]{inputenc}

\usepackage[a5paper, left=20mm, right=20mm, top=20mm, bottom=20mm]{geometry}

\begin{document}

\thispagestyle{empty}

\section*{Chaos macht Schule}

\textbf{Vortragende:} Marius Melzer \& Stephan Thamm, CCC Dresden\\
\textbf{Website:} http://c3d2.de\\
\\
``Alles was einmal im Internet ist, bleibt auch im Internet''
\subsection*{Datenschutzeinstellungen (im Firefox)}
\begin{itemize}
  \item Einstellungen -> Datenschutz -> Firefox wird eine Chronik: ``nach benutzerdefinierten Einstellungen anlegen''
    \begin{itemize}
      \item Häkchen bei ``Cookies von Drittanbietern akzeptieren'' entfernen
      \item ``Behalten, bis'' auf ``Firefox geschlossen wird'' stellen
    \end{itemize}
  \item Ghostery-Plugin (gegen Tracking)
  \item Adblock-Plus (gegen Werbung)
\end{itemize}
\subsection*{Sinnvolles Verhalten im Internet}
\begin{itemize}
  \item Möglichst nie den eigenen Namen verwenden
  \item Wenn sinnvoll, Wegwerf-Emailadressen verwenden (z.B. Mailinator oder 10minutemail)
  \item Immer unterschiedliche und sichere Passwörter (lang mit Sonderzeichen) wählen
  \item Privatsphäre-Einstellungen in Facebook und co. sollte man möglichst restriktiv setzen
  \item Datenschutz anderer respektieren, im Zweifel nachfragen!
\end{itemize}
\subsection*{Alternativen zu\ldots}
\begin{itemize}
  \item Google: ixquick, ecosia, duckduckgo,\ldots
  \item Facebook: Diaspora, Buddycloud,\ldots
\end{itemize}

\end{document}
